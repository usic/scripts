\large
    \setcounter{section}{0}
    \begin{center}
	\includegraphics[width=3 cm]{gerb} \\
        \huge \bf
        Угода з користувачем \input{LOGIN.tex} \\
        № \input{ID.tex}
    \end{center}
    \small
    \parbox{0.45\textwidth}{м.Київ}
    \parbox{0.45\textwidth}{\begin{flushright} \input{DAY.tex} \input{MONTH.tex} \input{YEAR.tex} р.  \\ \end{flushright}}

    \noindent Студентський інтернет-центр Національного університету „Києво-Могилянська академія“ (далі — Центр), який діє на підставі 
    відповідного Положення, в особі його керівника Синявського О.Л. та \textit{\input{FIO.tex}} (далі — Користувач) уклали   
    угоду про наступне:
%
    \section {Предмет угоди}
    \begin{enumerate}
        \item Забезпечення Центром Користувача інформаційними та освітніми послугами, а також додатковими неприбутковими послугами.
    \end{enumerate}
%
    \section {Обов’язки сторін}
    \begin{enumerate}
        \itemsep=-0.05 cm
        \item Центр бере на себе наступні обов’язки: 
        \begin{enumerate}
            \itemsep=-0.05 cm
            \item Безкоштовно надавати Користувачу інформаційні та освітні послуги, перелічені у статтях Регламенту та
            Положенні про студентський Інтернет-центр; 
            \item Здійснювати додаткові неприбуткові послуги; 
            \item Вчиняти інші дії, якщо це прямо передбачено \textit{Положенням}, \textit{Регламентом роботи}, іншими 
            нормативними документами, які регулюють діяльність Центру. 
        \end{enumerate}
        \item Користувач Центру зобов’язаний ознайомитися з цим \textit{Положенням}, \textit{Регламентом роботи} Центру, іншими нормативними документами, які регулюють діяльність Центру та неухильно їх дотримуватися. 
    \end{enumerate}
%
    \section {Обмеження щодо використання ресурсів Центру}
    \begin{enumerate}
        \item Користувач погоджується із заборонами та обмеженнями, встановленими у \textit{Положенні}.
    \end{enumerate}
%
    \section {Права Користувача}
    \begin{enumerate}
        \item Права Користувача визначені \textit{статтею 11 Положення}.
    \end{enumerate}
%
    \section {Відповідальність сторін}
    \begin{enumerate}
    \itemsep=-0.1 cm
    \item Жодна з умов цієї Угоди не може бути витлумачена у такий спосіб, що вона передбачає уникнення, зменшення чи
    збільшення юридичної відповідальності Сторін або інших осіб у разі, якщо згідно з чинними законами України існують
    правові підстави для її виникнення; 
    \item Межі відповідальності Центру та його співробітників визначені \textit{статтями 25 та 27 Положення}; 
    \item Межі відповідальності Користувача визначені \textit{статтею 26 Положення}.
    \end{enumerate}
%
    \section {Заключні положення}
    \begin{enumerate}
    \itemsep=-0.05 cm
    \item Особливості реєстрації Користувача визначає \textit{стаття 6 Регламенту роботи}; 
    \item Порядок набуття чинності, випадки та порядок втрати чинності цієї Угоди визначає \textit{стаття 7 Регламенту роботи}.
    \end{enumerate}
%
    \vfill
    \hspace {0.8\textwidth}
    \parbox{0.2\textwidth}{Користувач\\ \hrule}